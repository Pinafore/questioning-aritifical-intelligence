
The history of artificial intelligence begins in many ways with a
question answering game.  This idea came out of the research of a
researcher at the University of Manchester named Alan Turing.

\section{Turing's Legacy}
\label{sec:turing:legacy}

He’s probably better known for being one of the scientists who decoded
the Nazi Enigma device (which is why his memorial bench has ciphertext
underneath his name) and helped bring World War II in Europe to an
earlier conclusion.


\section{The Imitation Game}
\label{sec:turing:imitation}

But Turing is also a game designer.  Quoting from Bishop’s
description:
\begin{quote}
  Turing called for a human interrogator (C) to hold a conversation
  with a male and female respondent (A and B) with whom the
  interrogator could communicate only indirectly by typewritten text.
  The object of this game was for the interrogator to correctly
  identify the gender of the players (A and B) purely as a result of
  such textual interactions
\end{quote}

The players can lie, so the key to correctly deciding the genders of
the players is more about determining which player lacks key knowledge
about the experience of being a man or a woman.\footnote{What makes
  the Turing Test more poignant to me, at least, is that as a closeted
  gay man in a country where homosexuality is illegal, every
  interaction is dangerous: Turing is pretending to be something that
  he is not, and has to present himself as if he were a heterosexual
  man.  There's no evidence evidence to suggest that this was an
  inspiration for his first Turing Test, however.}

But what does this have to do with \abr{ai}?  Turing then thought,
let’s replace the man and the woman with a computer and a human.  If
the interrogator cannot determine which is the computer and which is
the machine, then the machine has displayed something that looks like
intelligence.

This is hard because there are lots of ``tells'' that can expose
someone who is transgressing against their assigned gender roles.
%
A skilled interrogator can ask about how it feels to have a period,
the rules for picking which urinal to use in a crowded restroom, or
how to respond to a girlfriend asking about her outfit.

And while an impostor can read many accounts to get an idea of what
someone has said about it, simply parroting these stories does not
equate to understanding.

\section{No, the Turing Test has not been Solved}
\label{sec:turing:failures}

\jbgcomment{Add CICERO rant here}

But I want to emphasize not just the broad strokes of the Turing Test,
but why we need to think critically about things that call themselves
the Turing test.  Also, it’s more fun than talking abstractly about
the Turing test.

Let’s start with \abr{parry}, a system designed by Kenneth Mark Colby
to simulate a paranoid schizophrenic.  And when you connected
psychiatrists to either real patients or \abr{parry}, they couldn’t
tell the difference.  So here the problem is the judges.  Not because
they aren’t experts--they are--but because their backgrounds prevent
them from being effective judges.

Psychiatrists are doctors bound by the hippocratic oath: they cannot
ask probing, in-depth questions that might harm a patient.  So, thus,
this really isn’t a Turing test.

Stu Schieber has a great take on this problem; I’d encourage you to
read the whole thing.  His take is on a competition called the Loebner
prize that purports to implement the Turing Test.  But here, the
interrogators are limited in the topics they can ask about.  So again,
competitions that don’t have skilled interrogators allowed to ask any
question they which.

\jbgcomment{Add Stu Schieber quote}

Another example that some people claim is an example of AI passing the
Turing Test is Google Duplex: Google offers to call a restaurant to
make a reservation for you.  Their text-to-speech system is very good,
but also puts in dysfluencies such as adding ``um'' and pauses to make
it sound more human.  Here, the judge is fooled into thinking that
they’re talking to a human.  But this doesn’t count either because the
judge doesn’t know they’re a judge!  The poor employee on the other
end of the phone call is expecting a human until proven wrong.

All of this doesn’t mean that the Turing Test is flawed.  It has
remained a part of AI for three quarters of a century because it’s a
simple, intuitive test of whether we have achieved artificial
intelligence.  So although we haven’t had a real Turing Test yet, a
judge asking questions of either an \abr{ai} or a human remains many
researchers’ goal.

\section{A Rigorous Test}
\label{sec:turing:test}

So let’s be true to the spirit of Turing’s idea of a parlor game.
Let’s make it visible to the public, let’s refine the rules and the
judges to make it more realistic and more fun.  By putting these games
in the public view and letting judges learn the best strategies for
discerning humans from computers, both sides can become worthier
adversaries.

And I think putting this slow advance of ever more capable computers
answering trickier questions should be out in front of the public.
Not just to keep the interrogators honest but to also keep the
companies and interests that sell AI honest.  The public has a vested
interest in knowing the limits of AI, and this is a fine way to make
that public.  But on the other side of the coin, it is also worthwhile
for the public to know when AI has really advanced… the public has a
right to know how computers react to challenging scenarios.  Better to
see them first played out for fun in a game than in high-stakes
transactions, a doctor’s office, or a courtroom.

But above all, this only works if we have good questions, so if we
believe that the Turing test really is the holy grail of AI, we as
humans need to know how to ask the right questions and computers need
to be able to answer any question that’s thrown at them.  In
Chapter~\ref{ch:gameshow}, we outline how this can become a gameshow.

\section{General Artificial Intelligence}
\label{sec:turing:gai}
