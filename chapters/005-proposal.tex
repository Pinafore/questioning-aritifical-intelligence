
\clearpage

\paragraph{Proposed Title and Subtitle} \vspace{3cm}

{\bf Questioning Artificial Intelligence}: How the Art of Asking Questions Defines AI and Helps Navigate an Uncertain Future

\vspace{3cm}


\paragraph{Full Description}



This book is not a complete introduction about how to build \abr{ai} systems
or how use natural language processing to extract answers from Wikipedia.
%
Nor is it an insider's guide to the world of high-stakes educational testing,
trivia games, or licensing exams.
%
We'll only introduce the reader to all of these interesting subcultures
from the trivial to the technical.

Unlike highly focused books on those individual domains, the goal of this book is to
draw \emph{connections} between the two worlds of human and computer
question answering, show how human expectations of \abr{qa} have
shaped \abr{ai} thus far, how there's still more to learn from
human \abr{qa}, and how a more mature \abr{ai} landscape may change
how humans answer questions.


\paragraph{Author Information}

I’m excited to be writing this because it’s something that I’m super
passionate about: as you’ll see very soon as we talk about the history of
question answering in artificial intelligence, answering questions is central
not just to my research program but also to my day job: teaching and guiding
students.  And it’s also important to as a hobby: even before this became part
of my research program, I was spending my weekends writing and answering silly
trivia questions.  I’m hoping that this course will help you appreciate the
joy and beauty of posing and answering questions.

In my day job, I have built question answering systems that used the structure
of Wikipedia to better answer questions, was one of the first to apply deep
learning to the \abr{qa} task, and played a lot of exhibition games against
trivia whizzes like Yogesh Raut, Ken Jennings, and Roger Craig.  I'm also a
professor of computer science, and one of the courses that I teach is a
course on question answering (and this book is the textbook for it).

As a hobby, I'm also a trivia enthusiast.  I played \qb{} for Caltech (where I
got \abr{bs} degrees in computer science and history), started Princeton's pub
quiz (where I went to grad school), and appeared on \jeopardy{}.
%
To be clear, I'm not very good at trivia: our \qb{} teams at best cracked top
ten nationally, and I came in second place in the one game of \jeopardy{} I
taped (my students are particularly fond of mocking my poor performance on the
video game category).
%
But this has brought me a familiarity with the trivia community and how it
works, one that I think gives me a unique (or deranged) look at how computers
answer questions.

In contrast, I've won more awards as a researcher.
%
Probably the one I'm most proud of is the Karen Sp\"ark Jones award (we'll
learn more about her in Chapter~\ref{}) from the British Computing Society,
but I've also gotten an \abr{nsf career} award and ``best of'' awards from Intelligent User
Interfaces, Neural Information Processing Systems, the North American
Association for Computational Linguistics, and the Conference on Natural
Language Learning.
%
I've published over a hundred peer-reviewed publications on interactive
machine learning, question answering, and exploring document collections; and
this isn't my first book, with David Mimno and Yuening Hu, I wrote
\emph{Applications of Topic Models}.
%
Although I should say that because I'm a professor, this is mostly the work of
the students that I've worked with over the years (I thank them by name at the
end of the book).

But the biggest prize that I've gotten is to have a career (and permissive
bosses) that allow me to combine my hobby with these intellectual pursuits.
%
And I'm even more lucky that that career allows me to now share it with you.

%%% Local Variables:
%%% mode: latex
%%% TeX-master: "../main"
%%% End:


\paragraph{Book Audience}

The book is intended either for the general public to learn about humand and computer
question answering or for machine learning researchers to learn
how \emph{human} question answering can improve their computer question
answering systems.

Possible classes where it could be used:
\begin{itemize}
        \item A freshman seminar introducing students to \abr{ai}.
        \item A graduate seminar in information seeking or evaluation.
        \item A companion to technical papers in a graduate computer science course.
\end{itemize}

\paragraph{Comparable Books}

\begin{itemize}
        \item Ken Jennings.  \textit{Braniac: Adventures in the Curious,
        Competitive, Compulsive World of Trivia Buffs}.  Villard, 2007.

        \item Mark T. Maybury.  \textit{New Directions in Question
        Answering}.  AAAI Press, 2004.

        \item Navin Sabharwal and Amit Agrawal.  \textit{Hands-on Question
        Answering Systems with BERT}.  A Press, 2021.
\end{itemize}


\paragraph{Additional Information and Specifications}

Estimated book contents:
\begin{itemize}
  \item 150 Single-spaced pages
  \item Two dozen diagrams / illustrations
  \item Draft to reviewers December 31, 2022, revisions/edits until
    June 31, 2021
\end{itemize}

Three chapters were published as position papers.  If accepted, the
chapters will be substantially revised to flow well with and
complement other chapters.

Sample chapter to follow.

\paragraph{How the Book is Structured}

We set the stage in two British Universities in the mid-twentieth
century and show how two researchers set the stage for rival paradigms
of artificial intelligence in the twentith century.
%
Having set the stage of this rivalry, we then trace the roots of these
two paradigms: civil service exams vs. library information desks, game
shows vs. standards bodies vs. standardized tests.

Having covered the history, the book turns to the present and the
hotly debated question of whether today's \abr{ai} is closer to Clever
Hans or to \abr{hal}.
%
We approach this central question through the
influence of the Cranfield and Manchester paradigms in three vignettes.
%
These stories detail three modern \abr{ai} triumphs where companies
have used question answering to turbocharge their \abr{ai}
research: \abr{ibm}'s Watson, Google's Natural Questions (and the connection to Muppets), and
Facebook's Dynabench leaderboard.
%
But rather than simply recount what these companies did, the
perspective of the first section allows us to critically examine the
important questions of whether the comparison between human and
machines is fair and how much the technical advantages reshaped the
field.

Finally, we turn our eyes to the future of \abr{ai} and \abr{qa}.
%
How can we take lessons from trivia games and standardized tests to know
when we've made progress in comparing human and machine intelligence
and to prod research in the right direction.

%%% Local Variables:
%%% mode: latex
%%% TeX-master: "../main"
%%% End:


