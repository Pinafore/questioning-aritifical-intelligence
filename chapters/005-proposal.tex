
\clearpage

\paragraph{Proposed Title and Core Message} \vspace{3cm}

{\bf Questioning Artificial Intelligence}: How the Art of Asking Questions Defines AI and Helps Navigate an Uncertain Future

\vspace{3cm}


\paragraph{Brief Blurb}

This book is not a complete introduction about how to build \abr{ai} systems
or how use natural language processing to extract answers from Wikipedia.
%
Nor is it an insider's guide to the world of high-stakes educational testing,
trivia games, or licensing exams.
%
But by introducing the reader to all of these places where questions
are being answered, this book helps introduce the reader to what it
takes for humans or computers to answer questions. 

By drawing those \emph{connections} between the two worlds of human
and computer question answering, this book shows how human
expectations of \abr{qa} have shaped \abr{ai}, how there's still more
to learn from human \abr{qa}, and how a more mature \abr{ai} landscape
may change how humans answer questions for fun and in the jobs of the
21\textsuperscript{st} century.

\paragraph{Keywords}

\begin{itemize}
\item Question Answering
\item Artificial Intelligence
\item Item Response Theory
  \item Natural Language Processing
\end{itemize}

\paragraph{Author Information}


I’m excited to write this because it’s something that I’m 
passionate about: answering questions is central
not just to my research program but also to my day job as a professor
at the University of Maryland teaching and guiding
students.  And it’s also important to me as a hobby: even before this became part
of my research program, I spent weekends writing and answering silly
trivia questions.  I’m hoping that this will help more people appreciate the
joy and beauty of posing and answering questions.

In my day job, I built question answering systems that used the structure
of Wikipedia to better answer questions~\citep{Zhao-21}, was one of the first to apply deep
learning to the \abr{qa} task~\citep{iyyer-15}, and played a lot of exhibition games against
trivia whizzes like Yogesh Raut,\footnote{https://www.youtube.com/watch?v=dyaR7zT_KKg} 
Ken Jennings,\footnote{https://www.youtube.com/watch?v=kTXJCEvCDYk} and Roger 
Craig.\footnote{https://www.youtube.com/watch?v=vH8cUGFOwPk}
I'm also a professor of computer science, and one of the courses that I teach is 
\textit{Good \abr{ai} Answers to Questions: How to Get Them and What Can Go Wrong} (and this book is the textbook for it).

I'm also a trivia enthusiast.  I played \qb{} for Caltech (where I
got \abr{bs} degrees in computer science and history), started Princeton's pub
quiz (where I went to grad school), and appeared on \jeopardy{} once as a contestant\footnote{https://www.j-archive.com/showgame.php?game_id=6112}
and once as a topic of discussion.\footnote{https://www.youtube.com/watch?v=q3HtpUjvYhA}
%
To be clear, I did not win many of those trivia games: our \qb{} teams at best cracked top
ten nationally, and I came in second place in the one game of \jeopardy{} I
taped (my students are particularly fond of mocking my poor performance on the
video game category).
%
But this has brought me a familiarity with the trivia community and how it
works, one that I think gives me a unique (or deranged) look at how computers
answer questions.

In contrast, I've won more awards as a researcher.
%
Probably the one I'm most proud of is the Karen Sp\"ark Jones award (we'll
learn more about her in Chapter~\ref{ch:ir}) from the British Computing Society,
but I've also gotten an \abr{nsf career} award and ``best of'' awards from Intelligent User
Interfaces, Neural Information Processing Systems, the North American
Association for Computational Linguistics, Empirical Methods in
Natural Language Processing, and the Conference on Natural
Language Learning.
%
I've published over a hundred peer-reviewed publications on interactive
machine learning, question answering, and exploring document collections, and
this isn't my first book, with David Mimno and Yuening Hu, I wrote
\emph{Applications of Topic Models}~\citep{boyd-graber-17}.
%
However, I should say that because I'm a professor, this is mostly the work of
the students that I've worked with over the years (I thank them by name at the
end of the book).

But the biggest prize I've earned is a career (and permissive
bosses) that allow me to combine my hobby with these intellectual pursuits.
%
And I'm even more lucky that that career allows me to now share it with you.

%%% Local Variables:
%%% mode: latex
%%% TeX-master: "../main"
%%% End:


\paragraph{Text Type and Market}

The book is intended either for the general public to learn about human and computer
question answering or for machine learning researchers to learn
how \emph{human} question answering can improve their computer question
answering systems.

Possible classes where it could be used:
\begin{itemize}
        \item A freshman seminar introducing students to \abr{ai}
          through the lens of question answering.
        \item A graduate seminar in information seeking or evaluation
          in information science.
        \item A companion to technical papers in a graduate computer
          science course.  And this is how I have used chapter drafts:
          as an overview to show the ``big picture'' of the technical
          papers we read.
\end{itemize}

The book is based on a course that I teach at the University of
Maryland.  The lectures are available online, and the most common
international visitors are from:
\begin{itemize}
\item India
\item China
  \item Germany
\end{itemize}

\paragraph{Competition}

\begin{itemize}
        \item Ken Jennings.  \textit{Braniac: Adventures in the Curious,
        Competitive, Compulsive World of Trivia Buffs}.  Villard, 2007.

        \item Mark T. Maybury.  \textit{New Directions in Question
        Answering}.  AAAI Press, 2004.

        \item Navin Sabharwal and Amit Agrawal.  \textit{Hands-on Question
        Answering Systems with BERT}.  A Press, 2021.
\end{itemize}

These books are different because they either focus on \emph{human}
question answering or \emph{computer} question answering, but not
drawing the connection between the two.

\paragraph{Length and Schedule}

Estimated book contents:
\begin{itemize}
  \item 150 Single-spaced pages
  \item Two dozen diagrams / illustrations
  \item Draft to reviewers January 31, 2025, revisions/edits until
    August 31, 2025
\end{itemize}

Three chapters were published as position papers.  If accepted, the
chapters will be substantially revised to flow well with and
complement other chapters.

Sample chapter to follow.

\paragraph{How the Book is Structured}

We set the stage in two British Universities in the mid-twentieth
century and show how two researchers set the stage for rival paradigms
of artificial intelligence in the twentith century.
%
Having set the stage of this rivalry, we then trace the roots of these
two paradigms: civil service exams vs. library information desks, game
shows vs. standards bodies vs. standardized tests.

Having covered the history, the book turns to the present and the
hotly debated question of whether today's \abr{ai} is closer to Clever
Hans or to \abr{hal}.
%
We approach this central question through the
influence of the Cranfield and Manchester paradigms in three vignettes.
%
These stories detail three modern \abr{ai} triumphs where companies
have used question answering to turbocharge their \abr{ai}
research: \abr{ibm}'s Watson, Google's Natural Questions (and the connection to Muppets), and
Facebook's Dynabench leaderboard.
%
But rather than simply recount what these companies did, the
perspective of the first section allows us to critically examine the
important questions of whether the comparison between human and
machines is fair and how much the technical advantages reshaped the
field.

Finally, we turn our eyes to the future of \abr{ai} and \abr{qa}.
%
How can we take lessons from trivia games and standardized tests to know
when we've made progress in comparing human and machine intelligence
and to prod research in the right direction.

%%% Local Variables:
%%% mode: latex
%%% TeX-master: "../main"
%%% End:


