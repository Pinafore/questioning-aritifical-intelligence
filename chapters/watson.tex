

I remember when I first heard rumblings of Watson.
%
Because I had a foot in both the \abr{ai} and trivia communities, I
heard two different stories.
%
I heard rumors of a amazon work in parsing and semantic role labeling
happening from researchers who ventured north to New York (I was doing
my PhD at Princeton).
%
From the trivia community, I heard of some people who were being paid
by \abr{ibm} to play trivia games but that they couldn't say anything
more.

I was very sceptical.
%
By the time that Watson campe to fruition, I had moved to the
University of Maryland.
%
Then, my scepticism turned to jealousy.
%
I watched, along with the rest of the world, one of the greatest
achievements of \abr{ai} unfold in front of me.
%
What was I doing wasting my time working on topic models if this was
also legitimate research?

Let me be clear that the technical triumphs are indisputable (and, in
my opinion, under-appreciated).
%
From the work on wagering to synthesizing multiple information
sources, Watson was from top to bottom a top-notch well-oiled machine.
%
And it computed all of this in real time---something that wasn't
strictly necessary but still impressive.

But where I have a criticism is with how the game itself was set up.
%
It's useful to go over the lifecycle of an entire question: how it was
written, how it's communicated, how players answer, and how the score
is tabulated.
%
At every stage, there's a slight benefit to the computer, which taken
together makes this an unfair competition.

This is a problem!  First, it's a problem scientifically because we
want to have fair comparisons of human vs. computer intelligence.
%
More importantly, I want to have my turn having my question answering
robots face off against trivia whizzes (Chapter~\ref{ch:game-show}),
and I can't do that if everybody thinks that Watson's spin
on \jeopardy{} settled the question (and they haven't).

% Cite / read this:
% https://dominoweb.draco.res.ibm.com/reports/rc25356.pdf

But first, in case you don't know how \jeopardy{} works, we'll review
that.
%
However, if you've calculated a Coryat score before, you can go ahead
and skip ahead.

\section{The Pool of Questions}

Part of the agreement between \jeopardy{} and \abr{ibm} was that the competition would take place on normal, written questions.
%
In the media coverage of the competition, this focused on avoiding video and picture daily doubles (fairly reasonable, but we'll discuss this more in a bit).
%


\section{Written, not Spoken}

\section{John Henry vs. the Buzzing Machine}

Not all trivia games with buzzers have this property, however.
For example, take \jeopardy{}, the subject of Watson's \textit{tour de force}~\cite{ferruci-10}.
While \jeopardy{} also uses signaling devices, these only work \emph{once the question has been read in its entirety}; Ken Jennings, one of the top \jeopardy{} players (and also a \qb{}er) explains it on a \textit{Planet Money} interview~\cite{malone-19}:
\begin{quote}
{\bf Jennings:} The buzzer is
    not live until Alex finishes reading the question. And if you buzz
    in before your buzzer goes live, \emph{you actually lock yourself out
    for a fraction of a second}. So the big mistake on the show is
    people who are all adrenalized and are buzzing too quickly, too
    eagerly. \\
{\bf Malone:} \abr{ok}. To some degree, \jeopardy{} is kind of a video game, and a \emph{crappy video game where it's, like, light goes on, press button}---that's it. \\
{\bf Jennings:} (Laughter) Yeah. \\
\end{quote}
\jeopardy{}'s buzzers are a gimmick to ensure good television; however, \qb{} buzzers discriminate knowledge (Section~\ref{sec:discriminative}).
Similarly, while \triviaqa{}~\cite{joshi-17} is written by knowledgeable writers, the questions are not pyramidal.

\section{Two Guys, One Computer}


\section{}
