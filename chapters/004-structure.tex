
While this book is obstensibly about how \emph{computers} answer questions,
that is not where the book begins.  After a brief introduction laying out the
organization of the book, we go to the \emph{history} of humans asking each other
questions.
%
This history is important because it explains why humans value answering
questions.
%
This provides historical context on why humans equate answering
questions correctly with intelligence but also because key elements for
artificial intelligence---item response theory and preference models---were
developed from looking at how humans answered questions.

The next stop is looking at \emph{today's} question answering, begining with
Alan Turing, arguably the father of computing and artificial intelligence.
%
Two wartime efforts---one at air and one at sea---began a rivalry in question
answering, which we call the Cranfield and Manchester based on the college
towns home to Alan Turing's theorizing on intelligent computing and Cyril
Cleverdon's practical evaluation of computing systems.
%
Having set the stage of this rivalry, we then see how these ideas built the
foundations of modern \abr{ai} from the methods that let \watson{} win on
\jeopardy{}
%
\ifproposal
(this is the subject of the sample chapter for the proposal)
\fi
%
to datasets like Google's Natural Questions that set the stage for
smart chatbots and the models like \abr{gpt}.

Finally, the book closes with the \emph{future} of question answering,
focusing on what it means for an \abr{ai} to be intelligent: how to turn Alan
Turing's thought experiment into reality and how to make sure that computers
don't trick us into thinking they're intelligent (but just cheating on the
test).
%
And if computers are becoming intelligent the question is how to \emph{align}
them with users, which sees the return of item response theory and preference
models: this time built up so we can better understand why \abr{ai} says the
things it does.
%
Finally, we close with a discussion of the ``big questions'' that computers
cannot answer on their own: how can the models developed from answering
questions from humans best work alongside humans for work and play.

%%% Local Variables:
%%% mode: latex
%%% TeX-master: "../main"
%%% End:
