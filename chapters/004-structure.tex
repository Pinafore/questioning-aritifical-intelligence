We set the stage in two British Universities in the mid-twentieth
century and show how two researchers set the stage for rival paradigms
of artificial intelligence in the twentith century.
%
Having set the stage of this rivalry, we then trace the roots of these
two paradigms: civil service exams vs. library information desks, game
shows vs. standards bodies vs. standardized tests.

Having covered the history, the book turns to the present and the
hotly debated question of whether today's \abr{ai} is closer to Clever
Hans or to \abr{hal}.
%
We approach this central question through the
influence of the Cranfield and Manchester paradigms in three vignettes.
%
These stories detail three modern \abr{ai} triumphs where companies
have used question answering to turbocharge their \abr{ai}
research: \abr{ibm}'s Watson, Google's Natural Questions (and the connection to Muppets), and
Facebook's Dynabench leaderboard.
%
But rather than simply recount what these companies did, the
perspective of the first section allows us to critically examine the
important questions of whether the comparison between human and
machines is fair and how much the technical advantages reshaped the
field.

Finally, we turn our eyes to the future of \abr{ai} and \abr{qa}.
%
How can we take lessons from trivia games and standardized tests to know
when we've made progress in comparing human and machine intelligence
and to prod research in the right direction.

%%% Local Variables:
%%% mode: latex
%%% TeX-master: "../main"
%%% End:
