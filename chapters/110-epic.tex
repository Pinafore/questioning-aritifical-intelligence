

Whith many tasks in computer science, asking the question ``where did this task come from'' has a simple answer usually in the form of an acronym . 
%
Character recognition was sparked by the \abr{mnist} dataset; entailment was sparked by \abr{snli}; sentiment classification was defined by \citet{}.

And while there are particular \emph{flavors} of question answering that habe been defined to make life easier or more discriminative for computers, the basic task of question in, answer out is natural to any inquisitive three year old.
%
Of course, question answering is not the only task like this: translation likewise is recognizable to any stranger in a strange land trying to cope with an unfamiliar language: dozens of books have been written on that subject, my favorite is \textit{Le Ton Beau de Marot}.

Question answering's roots likewise run deep.  In this chapter, I want to argue that question answering is not just a useful task but one that is woven deep into our Jungian subconsious.
%
Answering questions can define your identity, reveal fundamental truths of the universe, and create a shared reality.
%
This chapter highlights these pillars of question answering and how these questions of identity and knowledge shape not just our modern world but a world where our reality is also partially defined by artificial intelligence.

\section{The Sphynx}



In Greek myth, the Sphynx asked everyone who entered the city a riddle.  Here’s PDQ Bach’s interpretation of the riddle:

What starts out on four legs then goes round on two
Then finishes on three before it’s through

And the answer is:

Well a baby then a man, that’s plain, 
then finally an old geezer on a cane

https://www.youtube.com/watch?v=4sIp80L1dxY 8:30

Now I won’t spoil the whole story, but by answering the question correctly, Oedipus revealed that he was smarter than the average bear and also unlocked secrets from his past.  In a commencement address to Whitman college, Dana Burgess explained the difference between this and Google’s Natural questions and how that relates to the mission of professors (like me):

https://www.whitman.edu/academics/academic-calendar/convocation/convocation-history/convocation-2013/dana-burgess-the-gift-of-the-sphinx

Nobody told Oedipus who he was; he figured that out for himself.  ... You can do a Google search to find out the capital of Arkansas (Little Rock?) but you can’t do a Google search to find out who you are, what you’re good at, what makes you happy, what matters for your life.  Information transferal isn’t any help with that stuff.

When folks say that a college professor is like a sphinx, they don’t usually mean that in a nice way.  It suggests that we know the answers students need, but we keep them secret for the sake of our own power.  … That’s part of the riddle of [The Sphynx’s] hybrid nature: the curse of being forced to solve a riddle is also the gift of the ability to solve riddles. 

https://www.youtube.com/watch?v=2oZk6DWlw8I

% - Gestumblindi: the art of making a riddle is prized, and it is the
% Gods who are best at the task

% Perhaps add snow white and Turing's fascination with it? 