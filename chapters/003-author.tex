
I’m excited to be writing this because it’s something that I’m 
passionate about: answering questions is central
not just to my research program but also to my day job as a professor
at the University of Maryland: teaching and guiding
students.  And it’s also important to me as a hobby: even before this became part
of my research program, I spent weekends writing and answering silly
trivia questions.  I’m hoping that this will help more people appreciate the
joy and beauty of posing and answering questions.

In my day job, I have built question answering systems that used the structure
of Wikipedia to better answer questions, was one of the first to apply deep
learning to the \abr{qa} task, and played a lot of exhibition games against
trivia whizzes like Yogesh Raut, Ken Jennings, and Roger Craig.  I'm also a
professor of computer science, and one of the courses that I teach is a
course on question answering (and this book is the textbook for it).

As a hobby, I'm also a trivia enthusiast.  I played \qb{} for Caltech (where I
got \abr{bs} degrees in computer science and history), started Princeton's pub
quiz (where I went to grad school), and appeared on \jeopardy{}.
%
To be clear, I'm not very good at trivia: our \qb{} teams at best cracked top
ten nationally, and I came in second place in the one game of \jeopardy{} I
taped (my students are particularly fond of mocking my poor performance on the
video game category).
%
But this has brought me a familiarity with the trivia community and how it
works, one that I think gives me a unique (or deranged) look at how computers
answer questions.

In contrast, I've won more awards as a researcher.
%
Probably the one I'm most proud of is the Karen Sp\"ark Jones award (we'll
learn more about her in Chapter~\ref{ch:ir}) from the British Computing Society,
but I've also gotten an \abr{nsf career} award and ``best of'' awards from Intelligent User
Interfaces, Neural Information Processing Systems, the North American
Association for Computational Linguistics, Empirical Methods in
Natural Language Processing, and the Conference on Natural
Language Learning.
%
I've published over a hundred peer-reviewed publications on interactive
machine learning, question answering, and exploring document collections; and
this isn't my first book, with David Mimno and Yuening Hu, I wrote
\emph{Applications of Topic Models}.
%
Although I should say that because I'm a professor, this is mostly the work of
the students that I've worked with over the years (I thank them by name at the
end of the book).

But the biggest prize that I've gotten is to have a career (and permissive
bosses) that allow me to combine my hobby with these intellectual pursuits.
%
And I'm even more lucky that that career allows me to now share it with you.

%%% Local Variables:
%%% mode: latex
%%% TeX-master: "../main"
%%% End:
