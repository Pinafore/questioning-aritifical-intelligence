

% TODO(jbg): Cut, move to dynabench chapter

And this is not a criticism of \jeopardy{} authors: how should they
know how a computer thinks?
%
Working with a brilliant undergraduate, Eric Wallace, we built an
interface to show professional trivia writers what a computer is
thinking so that those writers could appropriately calibrate trivia
questions \emph{for a computer}~\citep{wallace-19}.

As authors type a question, they see the information the computer is
using to respond:
%
what Wikipedia articles it's using,
%
what previous clues it's looking at,
%
and what words in the clue it's using.
%
When professional trivia writers have access to this information, they can
avoid repeating themselves,
%
``Oh, thanks friendly popup, I didn't realize we just asked about
Grafton like that\dots I still want to ask about her, but let's mix
things up.''

My favorite example of how professional trivia writers used this
information to trip up computers was this clue:
\question{
  The main character of a story by this author opens Crime and
  Punishment to a random page, but finds it to be a copy of The
  Brothers Karamazov, and equates himself with Monsieur Bovary.
}
%
Now this is a hard clue (worthly of a competition between Ken and
Brad), and you may not know the answer; there's no shame in that.

But if you're relatively well-read, you probably know that the answer
is {\bf not} Fyodor Dostoevsky, the author of \textit{Crime and
  Punishment}.
%
How do you know that?  
%
First, Dostoyevski isn't really a self-referential author, so he
wouldn't write about his own novels\dots also, \textit{The Brothers
  Karamazov} was is last major work, so in what story could this
happen?
%
But while a computer technically ``knows'' all of this information, it
isn't going to use this to answer the question.

Instead, it sees the phrase ``this author opens Crime and Punishment''
and can pull up a dozen other questions have used this exact phrase,
it thus buzzes in confidently that the answer is \answer{Fyodor
  Dostoevsky}.
%
Indeed, you don't need to be a Dostoevsky expert to stay away from
that\dots you just need to understand a little linguistics.
%
The verb ``opens'' has the subject ``the main character'' \emph{not}
``this author''.
%
However, the computer greedily focuses on what it's seen before.

Again, this is not a criticism of the \jeopardy{} writers nor of the
\abr{ibm} folks who set up the competition.
%
These nuances are only obvious after the fact (and some of my early
human--computer competitions had similar shortcomings), but I hope
that it can help shape our future understanding of machine
intelligence.
%
We will talk about what happens when you set up a fair competition between
humans and computers on these types of questions in
Chapter~\ref{ch:leaderboard}.