We call someone smart if they can answer questions: the pass a test, win
on \jeopardy{}, or offer a witty retort in a debate.
%
This is not just a recent phenomenon.
%
Question answering is deeply embedded in
our culture:
%
in myth \OE{}dipus answered the riddle of the sphynx;
%
getting into a school depends on passing a standardized test;
%
good governance depends on civil service exams;
%
a trivia subculture has grown around answering all manner of questions.

\abr{ai} has been defined by asking questions even before it was ``a
thing'': Alan Turing designed a question answering proceduring to
determine if machine is intelligent.
%
In the years since, \abr{ai} has continued to be defined by questions:
answering questions on \jeopardy{} and on our phones.
%
From \abr{ibm} Watson on \jeopardy{} to Siri and Alexa in every home, the ability of
computers to answer is a common yardstick of how smart computers are.

As we’ll see this is important in both contemporary civilization and is
fundamental to artificial intelligence.  The goal of this book is to help
explain how computers answer questions on your phone, on game shows, and to
fight fake news.  Importantly, from my perspective, this won’t just be about
computers answering questions: humans and computers have a lot to learn from
each other, and I hope we’ll be able to discover new things about those
interactions.

This book looks at not just the \emph{how} of computers answers
questions but also \emph{why} answering questions is so important
for \abr{ai}.
%
Answering that ``why'' questions requires us to go back to
mythological, civil service exams, and game shows and connect them
to \abr{ai} question answering.

To be transparent, this book has an agenda: we can better understand
artificial intelligence by looking at its ability to answer questions
through this historical lens.
%
If we want to call a computer smart---indeed, smarter than a
human---we should make sure the competition is fair.
%
Moreover, writing and asking questions is an art that has been refined
over decades.

Part of the story I want to tell is how we got here.
%
And why little things like the decisions made at a tiny British
library in Bedfordshire fifty years ago have locked the artificial
intelligence community into a particular way of evaluating whether a
computer is smart.
%
These historical decisions have shaped the systems that we have today;
as a consequence, we are stuck with the crappy answers from our
smartphone.

But it didn't have to be this way!
%
The very definition of computing and artificial intelligence was based a
thought experiment that is closer to how computers and humans should interact
with each other.
%
However, that definition---from a boffin in Manchester called Alan
Turing---was too fanciful to really be implemented in the lab in the
twentieth century.
%
I'll argue that we could do something closer to Alan Turing's vision,
and you, the reader, can judge whether it's (still) too fanciful.

This is not just a narrow question for those building question
answering systems.
%
I think that the way we interact with our smartphones and computers
will define the future of human--computer interaction and
thus the shape of the modern, \abr{ai}-infused economy.


%%% Local Variables:
%%% mode: latex
%%% TeX-master: "../main"
%%% End:
